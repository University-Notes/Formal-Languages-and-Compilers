\documentclass{article}
\usepackage{algorithm,algorithmic}
\usepackage{mathtools}
\usepackage{dsfont}
\newcommand{\N}{\mathds{N}}
\usepackage{hyperref, titlesec}
\newcommand{\sectionbreak}{\clearpage}
\usepackage{algorithm,algorithmic}
\renewcommand{\algorithmiccomment}[1]{\hfill$\triangleright$\textit{#1}}
\usepackage{qtree}

\newcommand{\derives}[1][ ]{\xRightarrow{#1}}

\title{Formal Languages and Compilers}
\author{Matteo Secco}

\begin{document}
\maketitle\newpage
\tableofcontents\newpage

\section{Formal Language Theory}
\begin{description}
\item[Alphabet $\Sigma$:] any \underline{finite} set of symbols $\Sigma=\{a_1,a_2,...,a_k\}$
\item[String:] a sequence of alphabeth elements
\item[Language:] a set (possibly  infinite) of strings
\[
\Sigma=\{a,b,c\}
\quad
L_1=\{ab,ac\}
\quad
L_2=\{ab, aab, aaab, aaaab, ...\}
\]
\item[Sentences/Phrases:] strings belonging to a language
\item[Language cardinality:] number of sentences of the language
\[
|L_1|=|\{ab,ab\}|=2
\quad
|L_2|=|\{ab, aab, aaab, aaaab, ...\}|=\infty
\]
\item[Number of occurrences of a symbol in a string:] $|bbc|_b=2$, $|bbc|_a=0$
\item[Length of a string:] number of its elements
\[
|bbc|=3
\quad
|abbc|=4
\]
\item[String equality:] two strings $x=a_1a_2...a_h$ and $y=a_1a_2...a_k$ are equal $\iff$
\begin{itemize}
\item have same length: $|x|=|y| \iff h=k$
\item elements from left to  right coincide: $a_i=b_i \quad \forall i\in\{1..h\}$
\end{itemize}
\end{description}

\subsection{Operations on strings}
\begin{description}
\item[Concatenation] $x=a_1a_2...a_h \:\wedge \: y=b_1b_2...b_k  \implies  x\cdot y=a_1a_2...a_hb_1b_2...b_k$ 
	\begin{itemize}
	\item associative: $(xy)z=x(yz)$
	\item length: $|xy|=|x|+|y|$
	\end{itemize}
\item[Empty string] $\epsilon$ is the neutral element for concatenation: $x\epsilon = \epsilon x=x\:\forall x$.
	\begin{itemize}
	\item length: $|\epsilon|=0$
	\item \textbf{NB:} $\epsilon \neq \emptyset$
	\end{itemize}
\item[Substrings:] if $x=uyv$ then
	\begin{itemize}
	\item $y$ is a substring of $x$
	\item $y$ is a \underline{proper substring} of $x \quad \iff u\neq \epsilon \vee v\neq \epsilon$
	\item $u$ is a \underline{prefix} of $x$
	\item $v$ is a \underline{suffix} of $y$
	\end{itemize}
\item[Reflection:] if $x=a_1a_2...a_h$ then $x^R=a_ha_{h-1}...a_1$
	\begin{itemize}
	\item $(x^R)^R=x$
	\item $(xy)^R=y^Rx^R$
	\item $\epsilon^R=\epsilon$
	\end{itemize}
\item[Repetition:] $x^m=\underbrace{xxx...x}_{m\text{ times}}$. Inductive definition:
	\begin{itemize}
	\item $x^0=\epsilon$
	\item $x^m=x^{m-1}x \quad$ if $m>0$
	\end{itemize}
\end{description}
	
\subsection{Operations on Languages}
\begin{description}
\item[Reflection:] $L^R=\{ x|\exists y\left(y\in L \wedge x=y^R\right)\}$
\item[Prefixes(L):] $\{y|y\neq \epsilon \wedge \exists x \exists z \left(x \in L \wedge z \neq \epsilon \wedge x=yz \right)\}$
	\begin{itemize}
	\item \textbf{Prefix-free language: } $L \cap Prefixes(L)=\emptyset$
	\end{itemize}
\item[Concatenation:] $L'L''=\{xy|x \in L' \wedge y \in L''\}$
\item[Power:] inductive definition:
	\begin{itemize}
	\item $L^0=\{\epsilon\}$
	\item $L^m=L^{m-1}L	\quad$for $m>0$
	\item Consequences:
		\begin{itemize}
		\item $\emptyset ^0=\{\epsilon\}$
		\item $L \cdot \emptyset = \emptyset \cdot L = \emptyset$
		\item $L \cdot \{\epsilon\} = \{\epsilon\} \cdot L = L$
		\end{itemize}
	\end{itemize}
\item [Universal language:] over alphabet $\Sigma$: $L_{\text{universal}}=\Sigma^0 \cup \Sigma ^1 \cup ...$
\item[Complement:] of $L$ over $\Sigma$: $\neg L=L_{\text{universal}} \backslash L$
\item[Star:] formally called \textbf{reflexive and transitive closure} or \textbf{Klenee star} 
\[L^*=\bigcup_{h=0}^{\infty}L^h=L^0 \cup L^1 \cup ... = \epsilon \cup L^1 \cup L^2\]
\[\Sigma^*=L_{\text{universal}}\]
	\begin{description}
	\item[Monotonic:] $L \subseteq L^*$
	\item[Close under concatenation:] $x \in L^* \wedge y \in L^* \implies xy \in L^*$
	\item[Idempotent:] $(L^*)^*=L^*$
	\item[Commutative with reflection:] $(L^*)^R=(L^R)^*$
	\item $\emptyset^*=\{\epsilon\}$
	\item $\{\epsilon\}^*=\{\epsilon\}$
	\end{description}
\item[Cross:] $L^+=L \cdot L^*$
\item[Quotient:] $L_1 / L_2 =
\{y| \exists x \in L_1 \exists z \in L_2 (x=yz)\}$
	\begin{itemize}
	\item \textbf{Not set quotient!}
	\item Removes from $L_1$ suffixes contained in $L_2$
	\end{itemize}
\end{description}



\section{Regular Expressions and Languages}
\paragraph{Regular languages} are the simplest family of laguages.\\
They can be defined in three ways:
\begin{itemize}
\item Algebraically
\item Using generative grammars
\item Using recognizer automata
\end{itemize}
\subsection{Algebraic definition}
\paragraph{Regular expressions} are expression on languages that composes languages operations.\\
Formally
\begin{itemize}
\item Is a string $r$
\item Over the alphabet $\Sigma=\{a_1,a_2,...,a_n\} \cup \{\emptyset, \cup, \cdot, *\}$
\end{itemize}
Moreover, assuming $s$ and $t$ are regular expressions, then $r$ is a regular expression if any of the following rules applies:
\begin{itemize}
\item $r=\emptyset$
\item $r=a, \quad a \in \Sigma$
\item $r=s \cup t$ (alternative notation is $s|t$)
\item $r=s \cdot t$ (the $\cdot$ can be omitted)
\item $r=s^*$
\end{itemize}
\paragraph{The meaning} of a r.e. is a \textbf{language $L_r$} of alphabet $\Sigma$ according to the table:
\begin{table}[h!]
\begin{center}
\begin{tabular}{cc}
\textbf{Expression}	& 	\textbf{Language}\\
$\emptyset$			&	$\emptyset$\\
$\epsilon$			&	$\{\epsilon\}$\\
$a\in \Sigma$		&	$\{a\}$\\
$s \cup t$			&	$L_s \cup L_t$\\
$s \cdot t$			&	$L_s \cdot L_t$\\
$s^*$				&	$L_s^*$\\
\end{tabular}
\end{center}
\end{table}
\paragraph{Regular Languages} are languages denoted by a regular expression
\subsection{Language Families}
\paragraph{REG} is the collection of all regular languages
\paragraph{FIN} is the collection of all languages with finite cardinality
\subparagraph{Every finite language is regular} $FIN \subset REG$: 
\begin{itemize}
\item $L\in FIN \implies L = \bigcup_{i=1}^{k \in \N} x_i \implies L \in FIN$
\item $L=a^* \implies L \in REG \wedge L \notin FIN$
\end{itemize}



\subsection{Derivation}
\paragraph{Choice} Union and Concatenation corresponds to possible choices. One obtains subexpressions by making a choice that identifies a sub language.
\begin{table}[h!]
\begin{center}
\begin{tabular}{cc}
\textbf{Regular expression}	&	\textbf{Choices}\\
$e_1\cup ... \cup e_k$	&	$e_i \quad \forall i \in \{1,2,...,k\}$\\
$e^*$	&	$\epsilon$ or $e^n \quad \forall n\geq 1$\\
$e^+$	&	$e^n \quad \forall n\geq 1$\\ 
\end{tabular}
\end{center}
\end{table}
\paragraph{Derivation} among two r.e: $e_1 \derives e_2$ if 
\[e_1=\alpha \beta \gamma \wedge e_2=\alpha \delta \gamma\]
where $\gamma$ is a choice of $\beta$.\\
Derivation can be applied repeatedly, leading to $\derives[n]$ (deriving $n$ times, $\derives[*]$ (0 or more times), $\derives[+]$ (1 or more times). 
\paragraph{Language defined by an r.e.} $L(r)=\{x\in \Sigma^* | r \derives[*] x\}$
\paragraph{Equivalent r.e.} defines the same language


\subsection{Ambiguity of Regular Expressions}
\paragraph{Numbered subexpressions of a R.E}
\begin{itemize}
\item Add all passible parentheses to the r.e.
\item number the elements of $\Sigma$
\item identify all the subexpressions
\end{itemize}
\paragraph{Ambiguity} happens when a phrase can be obtained through distinct derivations, which differ \textbf{not only for the order}.
\paragraph{Sufficient condition for ambiguity} of the r.e. $f$ having numbered version $f'$ is that $\exists x \exists y \in L(f') | x \neq y$ but  $x=y$ when numbers are removed


\subsection{Extended Regular Expressions}
Regular expressions extended with other operators:
\begin{description}
\item[Power:] $a^n=\underbrace{aa...a}_{\text{n times}}$. NB: $n$ is an actual number, cannot be a parameter.
\item[Repetition:] from $k$ to $n>k$: $[a]_k^n=a^k \cup a^{k+1} \cup ... \cup a^n$
\item[Optionality:] $\epsilon \cup a$ or $[a]$
\item[Ordered interval:] $(0...9) \quad (a...z) \quad (A...Z)$
\item[Intersection]
\item[Difference]
\item[Complement]
\end{description}
It can be shown that Extended R.E. are not more powerful than standard R.E.
\paragraph{Closures} $REG$ is closed under
\begin{itemize}
\item Concatenation
\item Union
\item Star (*)
\item Cross (+)
\item Power
\item Intersection
\item Complement
\end{itemize}
\paragraph{Lists} contains an unspecified number of elements of the same type. Lists can be represanted with regex:
\[ie(se)*f\]
where $i$,$s$,$f$ are terminal symbols denoting the beginning of the list, a separator between elements, and the end of the string.
\subparagraph{Nested lists} are possible using regex if the nesting level is limited:
\begin{align*}
list_1 &= i_1 \cdot list_2 \cdot (s_1 \cdot list_2)^* \cdot f_1\\
list_2 &= i_2 \cdot list_3 \cdot (s_2 \cdot list_3)^* \cdot f_2\\
... &\\
list_k &= i_k \cdot e_k \cdot (s_k \cdot e_k)^* \cdot f_k
\end{align*}




\section{Context Free Grammars}
The language $L=\{a^nb^n|n>0\}$ is \textbf{not} regular.
\paragraph{Grammars} a tool to define language through \textbf{rewriting rules}. Phrases are generated hrough repeated application of the rules.
\paragraph{Context Free Grammar} is defined by 4 entities:
\begin{description}
\item[Non-terminal aplhabet] $V$
\item[Terminal alphabet] $\Sigma$, alphabeth of the resulting language
\item[Rules/Productions] $P$
\item[Axiom/Start] $S\in V$, from which derivation starts
\end{description}
\subparagraph{Rules form: }$X \rightarrow \alpha$ where $X \in V \wedge \alpha \in (V \cup \Sigma)^*$. Rules can be condensed:
\begin{align*}
X &\rightarrow \alpha_1\\
X &\rightarrow \alpha_2\\
...\\
X &\rightarrow \alpha_k\\
&\text{can be rewritten as}\\
X &\rightarrow \alpha_1 | \alpha_2 | ... | \alpha_k
\end{align*}.
\subparagraph{Safety conventions:}
\begin{itemize}
\item $\{\rightarrow, |, \cup, \epsilon\} \cap \Sigma = \emptyset$
\item $V \cap \Sigma = \emptyset$
\end{itemize}
\subparagraph{Notation conventions:}
$V$ elements can be distinguished using:
\begin{itemize}
\item $<$Angle brackets$>$ surrounding elements of $V$
\item Elements of $\Sigma$ in \textbf{bold}, elements of $V$ in \textit{italic}
\item Elements of $\Sigma$ 'quoted'
\item Elements of $V$ in UPPERCASE
\end{itemize}
\subsection{Types of rules}
\begin{description}
\item[Terminal] $\rightarrow u|\epsilon$
\item[Empty/Null] $\rightarrow \epsilon$
\item[Initial/Axiomatic] $S\rightarrow$
\item[Recursive] $A\rightarrow \alpha A \beta$
\item[Left-Recursive] $A \rightarrow A \beta$
\item[Right-Recursive] $A \rightarrow \alpha A$
\item[Left-and-Right-Recursive] $A \rightarrow A \beta A$
\item[Copy/Categorization] $A \rightarrow B$
\item[Linear] $\rightarrow uBv|w$
\item[Right-linear] $\rightarrow uB|w$
\item[Left-Linear] $\rightarrow Bv|w$
\item[Homogeneous normal] $\rightarrow A_1...A_n|a$
\item[Chomsky normal] $\rightarrow BC|a$
\item[Greibach normal] $\rightarrow a\sigma | b$ where $\sigma \in V^*$
\item[Operator normal] $\rightarrow AaB$
\subsection{Derivation}
\paragraph{Derivation $\implies$} Let $\beta, \gamma \in (V \cup \Sigma)^*$. Then $\beta \implies \gamma$ for grammar $G=<V,\Sigma, P, S>$ iff
\begin{align*}
\beta	&=\delta A \eta	&\wedge\\
A &\rightarrow \alpha \quad \in V &\wedge\\
\gamma &= \delta \alpha \eta
\end{align*}
\end{description}
Power, star and cross operators apply to derivation as usual %TODO formalize
\subsection{Erroneous Grammars and Useless Rules}
\paragraph{Clean grammar} $G=<V,\Sigma, P, S>$ is clean iff $\forall A \in V$
\begin{description}
\item[A is reachable:] $S \derives[*] \alpha A \beta$ where $\alpha , \beta \in (V \cap \Sigma)^*$
\item[A is defined:] $L_A(G)\neq \emptyset$ (generates a non-empty language)
\item[(G doesn't allow for circular derivations)] optional, but useful
\end{description}
\begin{algorithm}[h!]
\caption{Undefined nonterminals identification}
\begin{algorithmic}
\STATE $NEW \leftarrow \{A|(A\rightarrow u) \in P \wedge u \in \Sigma^*\}$
\REPEAT
\STATE $DEF \leftarrow NEW$
\STATE $NEW \leftarrow DEF \cup \{B | (B \rightarrow D_1D_2...D_n)\in P \wedge \overbrace{\forall i(D_i\in DEF \cup \Sigma)}^{D_i \text{in DEF or a terminal}}\}$
\UNTIL{$NEW=DEF$}
\STATE $UNDEF \leftarrow V \setminus DEF$
\end{algorithmic}
\end{algorithm}
\subparagraph{Produce relation} $A$ produce $B$ iff $A \rightarrow (\alpha B \beta)\in P$, where $A \neq B \wedge \alpha, \beta$ are strings
\begin{algorithm}[h!]
\caption{Unreachable nonterminals identification}
\begin{algorithmic}
\STATE Write the graph of the \textbf{produce} relation
\STATE Delete states that are not reachable from $S$
\end{algorithmic}
\end{algorithm}

\subsection{Infinite Languages and Recursion} 
Interesting languages are infinite. Infinite languages require the grammar generating them to be recursive
\paragraph{Recursive derivation} $A \derives[n] xAy$ 
\subparagraph{Immediately recursive derivation} $A \derives[1] xAy$ 
\subparagraph{Left-recursive derivation} $A \derives[n] Ay$ 
\subparagraph{Right-recursive derivation} $A \derives[n] xA$ 
\paragraph{Infinity condition} $|L(G)|=\infty \iff G$ is clean $\wedge$ $G$ avoids circular derivations $\wedge$ $G$ allows recursive derivations

\subsection{Syntax Trees and Canonical Derivations}
\paragraph{Syntax tree} A graph representing the derivation process which is
\begin{itemize}
\item Oriented
\item Sorted (Top-down, Left-to-right)
\item Acyclical
\item $\forall n_1, n_2 \exists!$ a path $n_1 \leftrightarrow n_2$
\end{itemize}
\paragraph{Subtree} with root $N$ is the tree having $N$ as root, includes $N$ and all its descendant
\paragraph{Example grammar}
\begin{itemize}
\item $E \to E + T | T$
\item $T \to T * F | F$
\item $F \to (E) | i$
\end{itemize}
\paragraph{Example sentence} $i+i*i$
\paragraph{Example tree} 
\Tree 
[.E_1 
	[.E_2 [.T_4 [.F_6 i ]]]
	+
	[.T_3
		[.T_4 [.F_6 i ]]
		*
		[.F_6 i ]
	]
]
				
\paragraph{Left derivation} the left-most rule is applied first
\paragraph{Right derivation} the right-most rule is applied first
\paragraph{Unicity of derivations} for a \underline{fixed} syntax tree $\exists!$ the left and right derivations
\paragraph{Skeleton tree} is equal to the syntax tree with all the non-terminals obscured
\Tree
[.$\bullet$ 
	[.$\bullet$ [.$\bullet$ [.$\bullet$ i ]]]
	+
	[.$\bullet$
		[.$\bullet$ [.$\bullet$ i ]]
		*
		[.$\bullet$ i ]
	]
]
\paragraph{Condensed skeleton tree} obtained from the skeleton tree by merging internal nodes and non-branching paths\\
\Tree
[.$\bullet$ 
	i
	+
	[.$\bullet$
		i
		*
		i
	]
]
\subsubsection{Parenthesis languages} Are expressed by the \underline{Dyck language}
\begin{itemize}
\item $\Sigma=\{a,c\}$
\item $S \to aScS | \epsilon$
\end{itemize}
We can observe that $L_1 = \{a^nc^n|n\geq 1\} \subset L_{\text{DYCK}}$


\subsection{Regular composition of (Context) Free Languages}
The family of free languages is closed under \underline{union, concatenation, kleen star}. Given $G_1=(\Sigma_1, V_{N_1}, P_1, S_1)$ and $G_2=(\Sigma_2, V_{N_2}, P_2, S_2$ such that $V_{N_1} \cap V_{N_2} = \emptyset \wedge S \notin (V_{N_1} \cup V_{N_2})$

\begin{description}
\item[Union] $G_1 \cup G_2=(\Sigma_1 \cup \Sigma_2, \quad V_{N_1} \cup V_{N_2} \cup \{S\}, \quad P_1 \cup P_2 \cup \underbrace{\{S \to S_1|S_2\}}_\text{execute concatenation at the beginning}, \quad S)$
\item[Concatenation] $G_1G_2=(\Sigma_1 \cup \Sigma_2, \quad \{S\} \cup V_{N_1} \cup V_{N_2}, \quad P_1 \cup P_2 \cup \underbrace{\{S \to S_1|S_2\}}_\text{choose one of the languages at the beginning}, \quad S)$
\item[Kleen star] $G_1^*=(\Sigma_1, \{S\} \cup V_{N_1}, \quad P_1 \cup \underbrace{ \{S\to SS_1|\epsilon \} }_\text{Perform repetition}, \quad S$
\item[Cross] $G_1^*=(\Sigma_1, \{S\} \cup V_{N_1}, \quad P_1 \cup \underbrace{\{S\to SS_1|S_1\}}_\text{Perform repetition}, \quad S$
\end{description} 
\subsection{Ambiguity}
\paragraph{Syntactic ambiguity} A sentence $x$ of a grammar $G$ is ambiguous if it admits multiple distinct syntax trees
\paragraph{Degree of ambiguity DOA} 
\begin{description}
\item[of a sentence ] $x$ of a language $L(G)$: $DOA(x)=$number of distinct trees for $x$ compatible with $G$
\item[of a grammar] $DOA(G)=max(\{DOA(x)|x \in L(G)\})$. It may happen that $DOA(G)=\infty$
\end{description}
\paragraph{Determining if a grammar is ambiguous} is a semi-decidable problem. Can be proven only if the grammar is ambiguous.
\subsubsection{Ambiguous forms and remedies}
\begin{description}
\item[Bilateral recursion] $S \to S x S|y$ where $x,y \in \Sigma \cup V$
	\subitem[Right-recursive] $S \to y S | y$
	\subitem[Left-recursive] $S \to S y | y$
\item[Left and right recursion in different rules] $S \to Sa|bS|c$
	\subitem[Separate] $S\to AcB$,
					$A\to Aa$,
					$B\to bB$
	\subitem[Enforce order] $S\to aS|B$,
							$B\to Xb|c$
\item[Union] If $G=G_1 \cup G_2$ and $L(G_1) \cap L(G_2) \neq \emptyset$ then some sentences in $G$ can be derived using both the rules of $G_1$ or the rules of $G_2$
	\subitem[Disjoint] provide disjointed set of rules: $G=(G_1\cap G_2) \cup (G_1 \setminus G_2) \cup (G_2 \setminus G_1)$ and the rules of these subsets are disjointed
\item[Concatenation] $G=G_1G_2$ is ambiguous if 
\[
\exists x_1,u \in L_1
\exists x_2,z \in L_2
\exists v \neq \epsilon|
x_1=uv \wedge x_2=vz
\]
\[
S \derives S_1S_2 \derives[+] uS_2 \derives[+] uvz \qquad \wedge \qquad
S \derives S_1S_2 \derives[+] uvS_2 \derives[+] uvz
\]
\item[Inherent ambiguity] $L$ is inherently ambiguous if any grammar $G$ for $L$ is ambiguous.
	\subitem[Avoidance] inherent ambiguity is rare and can be avoided
\item[Others] See slides %TODO
\end{description}
For practical purposes, it is also possible to modify the language (and for programming languages this may be desirable)
\subsection{Grammar equivalence}
\paragraph{Weak equivalence} $G_1$ and $G_2$ are weakly equivalent if $L(G_1)=L(G_2)$. Semi decidable
\paragraph{Strong equivalence} $G_1$ is strongly/structurally equivalent to $G_2$ if $L(G_1)=L(G_2)\wedge G_1$ and $G_2$ have the same \textbf{condensed skeleton tree}. Decidable
\subsection{Normal forms and Transformations}
\paragraph{Expansion of a non-terminal} allows to eliminate it from the rules where it appears
\[A\to xBy \quad B\to b_1|b_2|...|b_n\]
\[A\to xb_1y|xb_2y|...|xb_ny\]
\paragraph{Elimination of the axiom $S$ from right parts} obtained by introducing a replacement axiom:
\[S_{new}\to S_{old}\]
\subsubsection{Normal form without nullable nonterminals} Grammar such that $\forall A \in V\setminus \{S\} \quad \neg A\derives[+]\epsilon$
\paragraph{Nullable non-terminals}
$A$ is nullable if $\exists A \derives[+] \epsilon$
\subparagraph{Nullables set} $Null \subseteq V$
\begin{algorithm}[h!]
	\caption{Compute $Null$}
	\begin{algorithmic}
		\REPEAT
		\FORALL{$A \in V$}
		\IF{$A\to \epsilon \in P$}
		\STATE $A \in Null$
		\ELSIF{$A\to A_1A_2\dots A_n \in P|A_i \in V \setminus \{A\}$ \AND $\forall A_i(A_i \in Null)$}
		\STATE $A \in Null$
		\ENDIF
		\ENDFOR
		\UNTIL{convergence is reached}
	\end{algorithmic}
\end{algorithm}
\begin{algorithm}
\caption{Construction of non-nullable normal form}
\begin{algorithmic}
\STATE Compute $Null$
\FORALL{$R \in P$}
	\FORALL{$A \in Null$}
		\FORALL{Occurrencies of $A$ in the right part of $R$}
			\STATE Remove $A$ in the given occurrence of $R$
			\STATE Add the resulting rule to $P$
		\ENDFOR
	\ENDFOR
\ENDFOR
\FORALL{$R=A\to\epsilon\in P|A\neq S$}
	\STATE $P \leftarrow P\setminus R$
\ENDFOR
\STATE Clean the grammar
\STATE Remove circularities	
\end{algorithmic}
\end{algorithm}
\subsubsection{Copy rules elimination} Copy rules reduce grammar size but increase derivation length
\paragraph{Example: }$loop \to while | for | repeat$
\paragraph{Copy set} $Copy(A)=\{B \in V | \exists A \derives[*]B\}$
\begin{algorithm}
\caption{Computation of $Copy$}
\begin{algorithmic}
\REQUIRE $A \derives[+] \epsilon \implies A=S$ \COMMENT{no empty rules}
\REPEAT
	\FORALL{$A \in V$}
		\STATE $A \in Copy(A)$
		\FORALL{$B,C \in V$}
			\IF{$B \in Copy(A)$ \AND $B\to C\in P$}
				\STATE $C \in Copy(A)$
			\ENDIF
		\ENDFOR
	\ENDFOR
\UNTIL{convergence is reached}
\end{algorithmic}
\end{algorithm}
\begin{algorithm}
\caption{Definition of equivalent grammar without copy rules}
\begin{algorithmic}
\STATE $P' \leftarrow P \setminus \{A\to B|A,B \in V\}$ \COMMENT{delete copy rules}
\STATE $P' \leftarrow P' \cup \{A\to a|\exists B(B \in Copy(A) \wedge (B\to a)\in P)\}$ \COMMENT{add compensating rules}
\end{algorithmic}
\end{algorithm}
\subsubsection{Conversion from left to right recursion}
\paragraph{Immediate L-recursion}
\[
\begin{cases}
A\to AB_1|AB_2|\dots |AB_n\\
A\to a_1|a_2|\dots|a_h
\end{cases}
\]
becomes
\[
\begin{cases}
A\to a_1A'|a_2A'|\dots|a_hA'\\
A'\to B_1A'|B_2A'|\dots|B_na'
\end{cases}
\]
\paragraph{Non-immediate} not treated here
\subsubsection{Chomsky normal form} only 2 types of rules allowed:
\begin{itemize}
	\item $A\to BC$
	\item $A\to a$
\end{itemize}
\begin{algorithm}
\caption{Convert $G$  to Chomsky's normal}
\begin{algorithmic}
\STATE$P \leftarrow \emptyset$
\IF{$\epsilon \in L(G)$}
	\STATE $P \leftarrow \{S\to \epsilon\}$
\ENDIF
\FORALL{$R=x_0\to x_1x_2\dots x_n|x_i\in\Sigma\cup V$}
	\STATE $P \leftarrow P\cup \{x_0 \to X_1X_n\}$ \COMMENT{$X_1,X_n$ are newly created symbols}
	\STATE $P \leftarrow P \cup \{X_s \to x_2\dots x_n\}$
\ENDFOR
\IF{$x_1 \in \Sigma$}
	\STATE $P \leftarrow P \cup \{X_1 \to x_1\}$
\ENDIF
\end{algorithmic}
\end{algorithm}
\subsubsection{Real-time normal form} the right part of any rule has a terminal as prefix:
\[
A \to a\alpha|a \in \Sigma, \alpha \in \{\Sigma \cup V\}^*
\]
\subsubsection{Greibach normal form} special case of RT-nf: right parts are a terminal followed by 0 or more nonterminals
\[
A \to a\alpha|a \in \Sigma, \alpha \in V^*
\]



\subsection{CFG extensions and subsets}
\subsubsection{ENBF grammar (CFG+RE)} $G=\{V,\Sigma,P,S\}$. $|P|=|V|$. Rules are in the form $A\to \eta$, where $\eta$ is a RE over $V \cup \Sigma$
\subparagraph{Derivation} Given $\eta_1, \eta_2 \in (\Sigma \cup V)^*, \eta_1 \derives \eta_2$ if 
\begin{itemize}
	\item $\eta_1=\alpha A\gamma$
	\item $\eta_2=\alpha B \gamma$
	\item $A \to e \in P$ (e is an RE)
	\item $e \derives[*] B$
\end{itemize}

\end{document}